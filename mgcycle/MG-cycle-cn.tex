%!Mode:: "TeX:UTF-8"
%!TEX TS-program = xelatex

\documentclass[11pt,a4paper]{article}
\usepackage{fontspec,xunicode,xltxtra}
\usepackage[slantfont,boldfont]{xeCJK}

\usepackage{amsfonts}
\usepackage{latexsym}
\usepackage{amssymb}
\usepackage{graphicx}
%\usepackage{flafter}
\usepackage{xcolor}
\usepackage{float}
\usepackage[top=0.6in, bottom=0.6in, left=0.6in, right=0.6in]{geometry}
%
\usepackage[algo2e,ruled,linesnumbered,vlined]{algorithm2e}
\usepackage[usenames,dvipsnames]{pstricks}
\usepackage{pst-plot}
\usepackage{pst-node}
\usepackage{pst-math}

\pagestyle{plain}
\begin{document}

\begin{figure}[h!!]
\begin{center}
%
\psset{unit=0.75cm}

\begin{pspicture}(-0.3,-0.3)(12,6)
 \cnodeput[linecolor=black,fillstyle=solid,fillcolor=black](1.5,5){V2}{}
 \cnodeput[linecolor=black,fillstyle=solid,fillcolor=black](2.0,4){V3}{}
 \cnodeput[linecolor=black,fillstyle=solid,fillcolor=black](2.5,3){V4}{}
 \cnodeput[linecolor=black,fillstyle=solid,fillcolor=black](3.0,2.0){V5}{}
 \cnodeput[linecolor=black](3.5,1.0){V6}{}
 \cnodeput[linecolor=black,fillstyle=solid,fillcolor=black](4.0,2.0){V7}{}
 \cnodeput[linecolor=black,fillstyle=solid,fillcolor=black](4.5,3){V8}{}
 \cnodeput[linecolor=black,fillstyle=solid,fillcolor=black](5.0,4){V9}{}
 \cnodeput[linecolor=black,fillstyle=solid,fillcolor=black](5.5,5){V10}{}
%
\ncline[linecolor=blue, linewidth=0.04cm]{->}{V2}{V3}
\ncline[linecolor=blue, linewidth=0.04cm]{->}{V3}{V4}
\ncline[linecolor=blue, linewidth=0.04cm]{->}{V4}{V5}
\ncline[linecolor=blue, linewidth=0.04cm]{->}{V5}{V6}
\ncline[linecolor=red, linewidth=0.04cm]{->}{V6}{V7}
\ncline[linecolor=red, linewidth=0.04cm]{->}{V7}{V8}
\ncline[linecolor=red, linewidth=0.04cm]{->}{V8}{V9}
\ncline[linecolor=red, linewidth=0.04cm]{->}{V9}{V10}
\put(-0.4,4.9){细网格}
\put(-0.4,0.9){粗网格}

\cnodeput[linecolor=black,fillstyle=solid,fillcolor=black](8.0,3.9){V12}{} \put(8.6,3.8){\scriptsize 磨光}
\cnodeput[linecolor=black](8.0,3.1){V12}{} \put(8.6,2.95){\scriptsize 精确求解}
\psline[linecolor=blue, linewidth=0.04cm]{->}(7.8,2.6)(8.2,2.0) \put(8.6,2.1){\scriptsize 限制算子}
\psline[linecolor=red, linewidth=0.04cm]{->}(7.8,1.2)(8.2,1.8) \put(8.6,1.3){\scriptsize 提升算子}
\end{pspicture}
\vskip -0.7cm
  \caption{A schematic description of the V-cycle.} \label{fig_vcycle}
\end{center}
\end{figure}

\begin{figure}[H]
\begin{center}
%
\psset{unit=0.68cm}
\begin{pspicture}(0,0)(18,5.5)
 \cnodeput[linecolor=black](1,1.0){V6}{}
 \cnodeput[linecolor=black,fillstyle=solid,fillcolor=black](1.5,2.0){V7}{}
 \cnodeput[linecolor=black](2.0,1.0){V8}{}
 \cnodeput[linecolor=black,fillstyle=solid,fillcolor=black](2.5,2.0){V9}{}
 \cnodeput[linecolor=black,fillstyle=solid,fillcolor=black](3.0,3.0){V10}{}
 \cnodeput[linecolor=black,fillstyle=solid,fillcolor=black](3.5,2.0){V11}{}
 \cnodeput[linecolor=black](4.0,1.0){V12}{}
 \cnodeput[linecolor=black,fillstyle=solid,fillcolor=black](4.5,2.0){V13}{}
 \cnodeput[linecolor=black,fillstyle=solid,fillcolor=black](5.0,3.0){V14}{}
 \cnodeput[linecolor=black,fillstyle=solid,fillcolor=black](5.5,4.0){V15}{}
 \cnodeput[linecolor=black,fillstyle=solid,fillcolor=black](6.0,3.0){V16}{}
 \cnodeput[linecolor=black,fillstyle=solid,fillcolor=black](6.5,2.0){V17}{}
 \cnodeput[linecolor=black](7.0,1.0){V18}{}
 \cnodeput[linecolor=black,fillstyle=solid,fillcolor=black](7.5,2.0){V19}{}
\cnodeput[linecolor=black,fillstyle=solid,fillcolor=black](8.0,3.0){V20}{}
\cnodeput[linecolor=black,fillstyle=solid,fillcolor=black](8.5,4.0){V21}{}
\cnodeput[linecolor=black,fillstyle=solid,fillcolor=black](9.0,5.0){V22}{}
\cnodeput[linecolor=black,fillstyle=solid,fillcolor=black](9.5,4.0){V23}{}
\cnodeput[linecolor=black,fillstyle=solid,fillcolor=black](10,3.0){V24}{}
\cnodeput[linecolor=black,fillstyle=solid,fillcolor=black](10.5,2.0){V25}{}
\cnodeput[linecolor=black](11,1.0){V26}{}
\cnodeput[linecolor=black,fillstyle=solid,fillcolor=black](11.5,2.0){V27}{}
\cnodeput[linecolor=black,fillstyle=solid,fillcolor=black](12,3.0){V28}{}
\cnodeput[linecolor=black,fillstyle=solid,fillcolor=black](12.5,4.0){V29}{}
\cnodeput[linecolor=black,fillstyle=solid,fillcolor=black](13,5.0){V30}{}

\ncline[linecolor=red,linewidth=0.06cm,linestyle=dashed]{->}{V6}{V7}

\ncline[linecolor=blue,linewidth=0.04cm]{->}{V7}{V8}

\ncline[linecolor=red,linewidth=0.04cm]{->}{V8}{V9}
\ncline[linecolor=red,linewidth=0.06cm,linestyle=dashed]{->}{V9}{V10}

\ncline[linecolor=blue, linewidth=0.04cm]{->}{V10}{V11}
\ncline[linecolor=blue, linewidth=0.04cm]{->}{V11}{V12}

\ncline[linecolor=red, linewidth=0.04cm]{->}{V12}{V13}
\ncline[linecolor=red, linewidth=0.04cm]{->}{V13}{V14}
\ncline[linecolor=red,linewidth=0.06cm,linestyle=dashed]{->}{V14}{V15}

\ncline[linecolor=blue, linewidth=0.04cm]{->}{V15}{V16}
\ncline[linecolor=blue, linewidth=0.04cm]{->}{V16}{V17}
\ncline[linecolor=blue, linewidth=0.04cm]{->}{V17}{V18}

\ncline[linecolor=red, linewidth=0.04cm]{->}{V18}{V19}
\ncline[linecolor=red, linewidth=0.04cm]{->}{V19}{V20}
\ncline[linecolor=red, linewidth=0.04cm]{->}{V20}{V21}
\ncline[linecolor=red,linewidth=0.06cm,linestyle=dashed]{->}{V21}{V22}

\ncline[linecolor=blue, linewidth=0.04cm]{->}{V22}{V23}
\ncline[linecolor=blue, linewidth=0.04cm]{->}{V23}{V24}
\ncline[linecolor=blue, linewidth=0.04cm]{->}{V24}{V25}
\ncline[linecolor=blue, linewidth=0.04cm]{->}{V25}{V26}

\ncline[linecolor=red, linewidth=0.04cm]{->}{V26}{V27}
\ncline[linecolor=red, linewidth=0.04cm]{->}{V27}{V28}
\ncline[linecolor=red, linewidth=0.04cm]{->}{V28}{V29}
\ncline[linecolor=red, linewidth=0.04cm]{->}{V29}{V30}

\put(-1.2,4.9){细网格}
\put(-1.2,0.9){粗网格}

\cnodeput[linecolor=black,fillstyle=solid,fillcolor=black](14,4.4){V32}{}
\put(14.45,4.3){\scriptsize 磨光}
\cnodeput[linecolor=black](14,3.6){V32}{} \put(14.45,3.5){\scriptsize 精确求解}

\psline[linecolor=blue, linewidth=0.04cm]{->}(13.8,3.1)(14.2,2.5)
\put(14.45,2.7){\scriptsize 限制算子}

\psline[linecolor=red, linewidth=0.04cm]{->}(13.8,1.7)(14.2,2.3)
\put(14.45,1.9){\scriptsize 提升算子}

\psline[linecolor=red,linewidth=0.06cm,linestyle=dashed]{->}(13.8,0.9)(14.2,1.5)
\put(14.45,1.1){\scriptsize FMG 提升算子}

\end{pspicture}

%\vskip -0.7cm
  \caption{A schematic description of the full multigrid algorithm. } \label{fig_fmg}
\end{center}
\end{figure}

\begin{figure}[H]
\begin{center}
%
\begin{pspicture}(2,0)(15,4)
% V-cycle
\cnodeput*[linecolor=black, fillcolor=black](2.0,4.0){V1}{ }
\cnodeput*[linecolor=black, fillcolor=black](2.5,3.0){V2}{ }
\cnodeput*[linecolor=black, fillcolor=black](3.0,2.0){V3}{ }
%
\cnodeput[linecolor=black](3.5,1.0){V4}{ }
%
\cnodeput*[linecolor=black, fillcolor=black](4.0,2.0){V5}{ }
\cnodeput*[linecolor=black, fillcolor=black](4.5,3.0){V6}{ }
\cnodeput*[linecolor=black, fillcolor=black](5.0,4.0){V7}{ }
%
\ncline[linecolor=blue, linewidth=0.04cm]{->}{V1}{V2}
\ncline[linecolor=blue, linewidth=0.04cm]{->}{V2}{V3}
\ncline[linecolor=blue, linewidth=0.04cm]{->}{V3}{V4}
\ncline[linecolor=red, linewidth=0.04cm]{->}{V4}{V5}
\ncline[linecolor=red, linewidth=0.04cm]{->}{V5}{V6}
\ncline[linecolor=red, linewidth=0.04cm]{->}{V6}{V7}
%
\rput(3.6, 0.0){V--循环}
%
% W-cycle
\cnodeput*[linecolor=black, fillcolor=black](7.0,4){w1}{ }
\cnodeput*[linecolor=black, fillcolor=black](7.5,3){w2}{ }
\cnodeput*[linecolor=black, fillcolor=black](8.0,2.0){w3}{ }
%
\cnodeput[linecolor=black](8.5,1.0){w4}{ }
%
\cnodeput*[linecolor=black, fillcolor=black](9.0,2.0){w5}{ }
%
\cnodeput[linecolor=black](9.5,1.0){w6}{ }
%
\cnodeput*[linecolor=black, fillcolor=black](10.0,2.0){w7}{ }
\cnodeput*[linecolor=black, fillcolor=black](10.5,3.0){w8}{ }
\cnodeput*[linecolor=black, fillcolor=black](11.0,2.0){w9}{ }
%
\cnodeput[linecolor=black](11.5,1.0){w10}{ }
%
\cnodeput*[linecolor=black, fillcolor=black](12.0,2.0){w11}{ }
%
\cnodeput[linecolor=black](12.5,1.0){w12}{ }
%
\cnodeput*[linecolor=black, fillcolor=black](13.0,2.0){w13}{ }
\cnodeput*[linecolor=black, fillcolor=black](13.5,3.0){w14}{ }
\cnodeput*[linecolor=black, fillcolor=black](14.0,4.0){w15}{ }
%
\ncline[linecolor=blue, linewidth=0.04cm]{->}{w1}{w2}
\ncline[linecolor=blue, linewidth=0.04cm]{->}{w2}{w3}
\ncline[linecolor=blue, linewidth=0.04cm]{->}{w3}{w4}
\ncline[linecolor=red, linewidth=0.04cm]{->}{w4}{w5}
\ncline[linecolor=blue, linewidth=0.04cm]{->}{w5}{w6}
\ncline[linecolor=red, linewidth=0.04cm]{->}{w6}{w7}
\ncline[linecolor=red, linewidth=0.04cm]{->}{w7}{w8}
\ncline[linecolor=blue, linewidth=0.04cm]{->}{w8}{w9}
\ncline[linecolor=blue, linewidth=0.04cm]{->}{w9}{w10}
\ncline[linecolor=red, linewidth=0.04cm]{->}{w10}{w11}
\ncline[linecolor=blue, linewidth=0.04cm]{->}{w11}{w12}
\ncline[linecolor=red, linewidth=0.04cm]{->}{w12}{w13}
\ncline[linecolor=red, linewidth=0.04cm]{->}{w13}{w14}
\ncline[linecolor=red, linewidth=0.04cm]{->}{w14}{w15}
%
\rput(10.6, 0.0){W--循环}
%
\rput(1,4){细网格} %{×îÏž²ã}
\rput(1,1){粗网格}%{×îŽÖ²ã}
%\put(0,4.9){\scriptsize 细网格}
%\put(0,0.5){\scriptsize 粗网格}
%
\cnodeput*[linecolor=black, fillcolor=black](15.0,3.2){V16}{ }
\put(15.5,3.1){\scriptsize 磨光} %{Ä¥¹â}
\cnodeput[linecolor=black](15.0,2.4){V17}{ }
\put(15.5,2.3) {\scriptsize 精确求解} %{Ÿ«È·Çóœâ}
\psline[linecolor=blue, linewidth=0.04cm]{->}(14.8,1.9)(15.2,1.3)
\put(15.5,1.5){\scriptsize 限制算子} %{ÏÞÖÆ}
\psline[linecolor=red, linewidth=0.04cm]{->}(14.8,0.5)(15.2,1.1)
\put(15.5,0.7){\scriptsize 提升算子} %{ÌáÉýУÕý}
\end{pspicture}
%
  \caption{V-cycle and W-cycle} 
\end{center}
\end{figure}
%

\begin{figure}[H]
\begin{center}
%
\begin{pspicture}(3,0)(9,5)
%
\cnodeput[linecolor=black](3.5,1.0){V4}{ }
%
\cnodeput*[linecolor=black, fillcolor=black](4.0,2.0){V5}{ }
\cnodeput*[linecolor=black, fillcolor=black](4.5,3.0){V6}{ }
\cnodeput*[linecolor=black, fillcolor=black](5.0,4.0){V7}{ }
%
\ncline[linecolor=red, linewidth=0.04cm]{->}{V4}{V5}
\ncline[linecolor=red, linewidth=0.04cm]{->}{V5}{V6}
\ncline[linecolor=red, linewidth=0.04cm]{->}{V6}{V7}
%
\rput(3.6, 0.0){瀑布型多重网格}
%
\rput(2.5,4){细网格} %{×îÏž²ã}
\rput(2.5,1){粗网格}%{×îŽÖ²ã}
%\put(0,4.9){\scriptsize 细网格}
%\put(0,0.5){\scriptsize 粗网格}
%
\cnodeput*[linecolor=black, fillcolor=black](7.0,3.2){V16}{ }
\put(7.5,3.1){\scriptsize 磨光} %{Ä¥¹â}
\cnodeput[linecolor=black](7.0,2.4){V17}{ }
\put(7.5,2.3) {\scriptsize 精确求解} %{Ÿ«È·Çóœâ}
\psline[linecolor=red, linewidth=0.04cm]{->}(6.8,1.3)(7.2,1.90)
\put(7.5,1.5){\scriptsize 提升算子} %{ÌáÉýУÕý}
\end{pspicture}
  \caption{Cascadic multigrid} 
\end{center}
\end{figure}

\newpage

\begin{figure}[h!!]
%
\begin{center}
\unitlength=0.75mm
\begin{picture}(100, 60)(13,0)
\multiput(20, 42)(5, 0){9}{\line(1, 1){20}} %
\multiput(20, 42)(2.5, 2.5){9}{\line(1, 0){40}} %

\multiput(20, 21)(10, 0){5}{\line(1, 1){20}} %
\multiput(20, 21)(5, 5){5}{\line(1, 0){40}} %

\multiput(20, 0)(20, 0){3}{\line(1, 1){20}} %
\multiput(20, 0)(10, 10){3}{\line(1, 0){40}} %

\thicklines
\put(12, 16){\textcolor{red}{\vector(0, 1){25}}}
\put(-12, 28){\textcolor{red}{{提升算子}}}
\put(88, 43){\textcolor{blue}{\vector(0, -1){25}}}
\put(92, 28){\textcolor{blue}{{限制算子}}}
\put(85, 5){粗网格}
\put(85, 52){细网格}
\end{picture}
\end{center}
\vskip -0.5cm
%
\caption{Pictorial demo of multigrid method with three levels of grids.} \label{fig_multigrid}
\end{figure}

\end{document}
